W dwa lata po Stanisławie przyszła na świat Józefa Głąbówna (ur. 28 VIII 1904 r. w Mirowie).

%TODO ***Tu zdj. aktu urodz. Józefy Głąb

Wyszła ona 8 lutego 1932 r. za Józefa Machurę (ur. 8 III 1908 r. w Górze Włodowskiej). Wesele odbyło się w domu Panny Młodej, tj. w Mirowie, na którym trzydziestoczteroletni kawaler Jan Głąb -- brat ciotki Józki poznał młodszą siostrę Pana Młodego -- Stanisławę Machurzankę i w niej się zakochał, co finał swój znalazło na weselu -- tym razem we Włodowicach i w Górze 5 IX 1932 r.

%TODO ***Tu zdj. Józefa i Józefy Machurów z dziećmi}

Miała z nim trójkę dzieci: syna Edwarda (ur. 25 IV 1934 r. w Górze Włodowskiej), który zginął w wypadku motocyklowym 5 VIII 1959 r., córkę Marię Irenę (ur.28 IV 1937 r. w Górze Włodowskiej) oraz Genowefę (ur. 2 II 1942 r. w Górze Włodowskiej).

%TODO *** Tu zdj. Marii Kułach z córkami}

Córka Maria wyszła dnia 5 II 1961 r. we Włodowicach za Ryszarda Józefa Kułacha (ur. 17 III 1931~r. w Górze Włodowskiej z ojca Bartłomieja i matki Ludwiki z domu Biała), z którym ma dwie córki: Krystynę (ur. 5 II 1961 r. w Żarkach) i Małgorzatę (ur. 15 VII 1970 r. w Myszkowie). Ryszard Kułach wielokrotnie wchodził w kolizję z kodeksem karnym więc sporą część życia spędził w więzieniu, a o matkę z córkami nie dbał. Mieszkają one, jak matka, w Myszkowie. Krystyna wyszła za Adama Szewczyka 
%TODO znaleźć datę (ur. dnia ...... 19.. r. w Myszkowie) 
i miała z nim córkę Martę (ur. 19 VI 1985 r. w Żarkach) oraz syna Marcina (ur. 28 VII 1987 r. w Myszkowie).

%TODO {\color{red} *** Tu zdj. Krystyny i Adama Szewczyków z dziećmi}

Krystyna Kułach Szewczyk zmarła 25 I 2007 r. w Myszkowie w wieku 46 lat.

%TODO {\color{red} *** Tu zdj. Małgorzaty i Roberta Morawskich z córkami.}

Małgorzata Kułach wyszła za Roberta Morawskiego (ur. 7 VII 1968 r. w Myszkowie z ojca Stanisława i matki Małgorzaty z domu Bramora) i ma z nim dwie córki: Klaudię Morawską (ur. 17 IV 1988 r. w Myszkowie) oraz Magdalenę Morawską (ur. 24 VII 1993 r. w Myszkowie). 

Młodsza  córka Józefa i Józefy Machurów -- Genowefa wyszła dnia 13 lipca 1969 r. za Aleksandra Grelę (ur. 15 I 1944 r.), z którym miała trójkę dzieci: syna Pawła (ur. 1970 r. w Górze Włodowskiej), córkę Katarzynę (ur. 1972 r. w Górze Włodowskiej) syna Marcina (ur. 1976 r. w Górze Włodowskiej, który zginął tragicznie w 1994 r. porażony prądem w warsztacie ojca Aleksandra, z czego ten do dziś nie może się otrząsnąć.

%TODO {\color{red} *** Tu zdj. Genowefy i Aleksandra Grelów z dziećmi}

Do tego nieszczęścia w niedługim czasie dołączyło drugie, to jest bankructwo jego firmy spowodowane niewypłacalnością firmy państwowej, dla której wykonał około tysiąca metrów łańcucha galla za cenę 400 tys. PLN, przy czym sam materiał kosztował go około 200 tys. PLN. Zanim wygrał sprawę w sądzie już go zlicytowano!
